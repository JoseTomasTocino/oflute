\begin{frame}{Analizador básico}
  \begin{block}{Objetivo}
    Desarrollar un módulo que capture el sonido del micrófono, lo analice y
    detecte la nota que se está tocando.
  \end{block}

  \pause

  \begin{block}{Primer paso: capturar el audio}
    \begin{itemize}
    \item Se utilizó la API de PulseAudio.
    \item Abrimos un flujo de entrada.
    \item Creamos un búffer para recoger los datos.
    \item Procesamos los datos cuando se llena el búffer.
    \end{itemize}
    
  \end{block}
\end{frame}

\begin{frame}{Analizador básico}
  \begin{block}{Segundo paso: analizar el sonido}
    \begin{itemize}
    \item Trabajamos con el contenido del búffer.
    \item Aplicamos el algoritmo DFT.
    \item Aislamos la frecuencia fundamental.
    \item Comparamos la frecuencia fundamental con una tabla de frecuencias para
      la flauta dulce.
    \item Devolvemos la nota detectada.
    \end{itemize}
  \end{block}  
\end{frame}

\begin{frame}[fragile]{Carga de fuentes TrueType}
  \begin{center}
    oFlute utiliza \textbf{Gosu} como sistema gráfico.
    
    \pause
    \medskip
    
    \textcolor{red}{\textbf{Problema:}} Gosu no permite cargar fuentes TrueType
    en GNU/Linux.

    \pause
    \medskip
    
    \textcolor{dgreen}{\textbf{Solución:}} se implementa un módulo propio para
    carga y pintado de fuentes TrueType. \pause\\[1em]Este módulo se liberó y pasó a
    formar \textbf{parte oficial} de Gosu.
  \end{center}
  \begin{minted}{cpp}
// Used for custom TTF files
// Adapted from customFont class by Jose Tomas Tocino Garcia (TheOm3ga)
class SDLTTFRenderer : boost::noncopyable
  \end{minted}
\end{frame}

\begin{frame}{Animaciones dinámicas}
  \begin{center}
    \textcolor{red}{\textbf{Problema:}} uno de los objetivos era tener
    interfaces amigables, fluidas y minimalistas.

    \pause \medskip

    \textcolor{dgreen}{\textbf{Solución:}} se desarrolla un sistema de
    animaciones mediante interpolaciones de movimiento.

    \pause\medskip

    Permite movimientos de aceleración, deceleración, uniformes, etcétera. Es
    extensible a un número arbitrario de atributos.

    \pause\medskip

    Se basó en las ecuaciones de Robert Penner, \\liberadas bajo licencia BSD.
    
  \end{center}  
\end{frame}

%%% Local Variables: 
%%% mode: latex
%%% TeX-master: "../presentacion"
%%% End: 
