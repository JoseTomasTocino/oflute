
\begin{frame}{Contexto social}
  \begin{center}
    \Large

    Jóvenes en plena simbiosis con las nuevas tecnologías.

    \bigskip
    \pause
    \bigskip

    Las TIC están llegando a los centros educativos.

    \bigskip
    \pause
    \bigskip

    Técnicas docentes basadas en recursos multimedia e informáticos.

  \end{center}
\end{frame}

\begin{frame}{Concepción del proyecto}
  \begin{block}{Idea}
    Hacer un juego educativo.
  \end{block}

  \pause

  \begin{block}{Primera cuestión}
    ¿Sobre qué aspecto educativo? ¿Qué asignatura se beneficia?
  \end{block}

  \vspace{0.5cm}
  \pause

  \begin{center}
    \LARGE Música \\[0.5cm]

    \pause

    \LARGE Aprendizaje de la flauta dulce
  \end{center}
\end{frame}

\begin{frame}{Objetivos}
  \Large
  \begin{itemize}
  \item Adquisición de base de conocimientos. \pause
  \item Módulo de análisis de sonido. \pause
  \item Sistema de interpretación de canciones. \pause
  \item Sistema de lecciones ampliable. \pause
  \item Interfaz de usuario amigable y fluida.
  \end{itemize}
\end{frame}

\begin{frame}{Motivaciones personales}
  \begin{itemize}
  \item Representación digital del sonido.
  \item Programación de audio.\pause

  \item Bases del DSP (Procesamiento digital de señales).
  \item Técnicas básicas de análisis de audio.\pause

  \item Ampliar conocimientos sobre desarrollo de videojuegos.
  \item Aprender nuevas tecnologías.
  \item Aportar al software libre.
  \end{itemize}
\end{frame}

%%% Local Variables: 
%%% mode: latex
%%% TeX-master: "../presentacion"
%%% End: 
