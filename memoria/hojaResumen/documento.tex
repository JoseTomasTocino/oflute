\documentclass{article}

%\usepackage[utf8]{inputenc}
\usepackage[spanish]{babel}

\usepackage{graphicx}
\usepackage{hyperref}
\usepackage{fancyvrb}
\usepackage{caption}

\usepackage{fontspec}   % provides font selecting commands 
\usepackage{xunicode}   % provides unicode character macros 
\usepackage{xltxtra}    % provides some fixes/extras
\setmonofont{Consolas}
\setsansfont{Calibri}

\usepackage{setspace}
\onehalfspacing
%\doublespacing

\setlength{\parskip}{0.15cm}
%\setlength{\headsep}{0pt}


%

\usepackage[compact]{titlesec}
\titlespacing{\section}{0pt}{*0}{*0}
\titlespacing{\subsection}{0pt}{*0}{*0}
\titlespacing{\subsubsection}{0pt}{*0}{*0}

\DeclareCaptionStyle{base}[justification=centering,indention=0pt, margin={5mm,10mm,15mm,20mm}]{}

% Véase http://www.andy-roberts.net/misc/latex/latextutorial6.html , sección custom floats
\usepackage{float}
\floatstyle{ruled}
\newfloat{code}{th}{lop}
\floatname{code}{Código}
\newfloat{code*}{th}{lop}

\usepackage[top=2cm, bottom=2cm, left=2cm, right=2cm]{geometry}

%% \usepackage{multicol}
%% \setlength{\columnsep}{20pt}

\renewcommand{\familydefault}{\sfdefault}

\title{oFlute: Resumen del proyecto}
\author{José Tomás Tocino García}
\date{}

\begin{document}

\pagestyle{empty}
\begin{center}
\begin{Large}\textbf{oFlute: Resumen del proyecto}\end{Large}\\
\begin{large}José Tomás Tocino García - \texttt{<theom3ga@gmail.com>}\end{large}\\[0.1cm]

% {\small Licencia CC - Reconocimiento - No comercial - Compartir igual.\\
% http://creativecommons.org/licenses/by-nc-sa/2.5/es/}
 
\end{center}

\part*{Resumen}

El proyecto comprende dos etapas. Primero, el aprendizaje autodidacta,
y la posterior aplicación del conocimiento adquirido, en el diseño y
desarrollo de un módulo de análisis de sonidos destinado a capturar,
mediante un micrófono, el sonido emitido por una flauta en tiempo
real, y calcular al vuelo qué nota está siendo reproducida,
permitiendo al sistema responder según la misma.

La segunda parte del proyecto es una aplicación del módulo
anteriormente citado. Así, \textbf{oFlute} es una aplicación
lúdico-educativa pensada principalmente para el alumnado de educación
primaria que comienza a aprender a usar la flauta dulce,
proporcionando un entorno atractivo y ameno para el estudiante.

Esta aplicación se compone de varias partes:
\begin{itemize}
\item Una sección de \textbf{análisis básico de notas}, en las que el usuario
  podrá comprobar con tranquilidad la representación de las diferentes
  notas que toca con su flauta en un pentagramas que se actualizará
  dinámicamente en pantalla.
\item Un \textbf{motor de lecciones}, que presenta una serie de unidades
  didácticas en formato multimedia, compuestas de imágenes y textos,
  con conceptos sobre música, manejo de la flauta, etc. Este sistema
  es totalmente dinámico y fácilmente ampliable.
\item Un sencillo sistema de \textbf{calibración del micrófono}, con el que
  medir el ruido de ambiente y así ajustar el sistema a diferentes
  entornos.
\item Un \textbf{motor de canciones}, al más puro estilo
  SingStar\textregistered \footnote{SingStar\textregistered es una
    serie de juegos de karaoke en los que el jugador debe cantar con
    la música siguiendo unas barras en pantalla que indican el tono a
    entonar.}, que permite al usuario practicar sus habilidades con la
  flauta siguiendo un pentagrama en pantalla. Se pueden añadir
  canciones fácilmente.
\end{itemize}

A consecuencia del desarrollo del proyecto se han generado
paralelamente una serie de \textbf{valores añadidos}:
\begin{itemize}
\item Los conocimientos adquiridos sobre la biblioteca
  Boost\footnote{\url{http://www.boost.org}} hicieron posible la
  \textbf{celebración del taller de Boost} celebrado el día 30 de
  Junio de 2010 en el marco de los Cursos de Verano 2010, sitos en la
  Escuela Superior de Ingeniería de Cádiz y organizados por la OSLUCA.
\item Los conocimientos sobre la biblioteca de desarrollo de
  videojuegos Gosu favorecieron la creación y el desarrollo del
  proyecto personal \textbf{Freegemas}
  \footnote{\url{http://code.google.com/p/freegemas}}, un clon open source,
  multiplataforma, del clásico juego Bejeweled.
\item También se desarrolló un sistema para poder utilizar fuentes
  truetype en Gosu bajo sistemas GNU/Linux. Este sistema finalmente se
  incluyó \textbf{de manera oficial} en la biblioteca Gosu.
\item Por último, otras partes del proyecto tienen valor de forma
  independiente, como el \textbf{sistema de animaciones}, que está
  completamente desligado del resto del proyecto y puede usarse con
  cualquier otro sistema, o las \textbf{utilidades de logging}, que
  siguiendo la filosofía del software libre han sido reutilizadas y
  ampliadas por otros desarrolladores (en este caso por Fabián Sellés
  Rosa).
\end{itemize}

\end{document}
