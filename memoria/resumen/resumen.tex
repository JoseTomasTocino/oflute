\documentclass[a4paper,11pt]{article}

\usepackage{estiloBase}
\usepackage{colores}
\usepackage{bera}
\usepackage{comandos}


\def \titulo{oFlute: blablablá título largo}
\def \autor{Alumno: José Tomás Tocino García\\Tutores: Manuel Palomo Duarte, Antonio García Domínguez}
\def \fecha{Agosto de 2010}

%\margenes

% Directorio de imágenes
%\graphicspath{{../img/}}

\begin{document}
\portada

\abstract{\textbf{oFlute} se modela como una herramienta lúdico-educativa para
  alumnos que comiencen a aprender a usar la flauta dulce, proporcionando un
  entorno atractivo y ameno para el estudiante. Éstos tendrán la posibilidad de
  comprobar sus conocimientos sobre el uso de la flauta de forma totalmente
  práctica, gracias a un motor de análisis del sonido capaz de detectar las
  notas que emite el jugador con la flauta, capturadas por un micrófono,
  mediante el que la aplicación valorará la pericia del estudiante con la
  flauta.

  Además, los jugadores podrán recorrer una serie de pequeñas lecciones sobre
  música en general, y el uso de la flauta dulce en particular. Estas lecciones
  son totalmente ampliables, dando al usuario la posibilidad de crear las suyas
  propias. }

\vspace{0.5cm}

\begin{center}
{\footnotesize Este documento se halla bajo la licencia FDL de GNU (Free Documentation
  License)\\ \url{http://www.gnu.org/licenses/fdl.html} }   
\end{center}



\tableofcontents

\lstset{style=C++}

%\setlength{\parskip}{0.3cm plus 3mm}
\setlength{\parindent}{0.3cm}

\section{Introducción}

\subsection{Contexto y motivación}
Las nuevas tecnologías van filtrándose gradualmente en los centros
educativos, y las técnicas de enseñanza se están adaptando a las
opciones que ofrecen. El reparto de ordenadores portátiles a los
alumnos andaluces de 5º y 6º de primaria, dentro del marco de la
Escuela TIC 2.0, es buena muestra de ello. 

Por otro lado, las nuevas generaciones están en plena simbiosis con las
tecnologías de la información, cada vez más acostumbradas al empleo de
dispositivos electrónicos interactivos, y su uso ya les es prácticamente
instintivo. Por tanto, es beneficioso buscar nuevos métodos educativos que hagan
uso de las nuevas tecnologías.

En la búsqueda de materias educativas en las que aplicar el uso de las nuevas
tecnologías, la música, parte fundamental del programa curricular en la
educación primaria, ofrece una gran variedad de aspectos que podrían
desarrollarse utilizando tecnologías de la información. Es ahí donde este
proyecto hace su aportación, en la flauta dulce, un instrumento económico y
fácil de aprender que se usa tradicionalmente en la educación musical
obligatoria en España.

\subsection{Objetivos}
Los principales objetivos a alcanzar con \textbf{oFlute} son los siguientes:

\begin{itemize}
\item Crear un módulo de análisis del sonido en el dominio de la frecuencia para
  poder identificar las notas emitidas por una flauta dulce y capturadas
  mediante un micrófono en tiempo real.
\item Crear una aplicación de usuario que identifique y muestre en pantalla las
  notas que toca el usuario con la flauta dulce en cada momento.
\item Reutilizar el módulo de análisis en un juego en el que el
  usuario debe tocar correctamente las notas que aparecen en pantalla
  siguiendo un pentagrama.
\item Incluir un sistema de lecciones multimedia individuales que
  sirvan al alumno de referencia y fuente de aprendizaje.
\item Potenciar el uso de interfaces de usuario amigables, con un
  sistema avanzado de animaciones que proporcione un aspecto fluido y
  evite saltos bruscos entre secciones.
\item Obtener una base teórica sobre cómo se representa y caracteriza
  digitalmente el sonido.
\item Conocer las bases del DSP, y su uso en aplicaciones de
  reconocimiento básico de sonidos, tales como sintonizadores y
  afinadores de instrumentos.
\item Adquirir soltura en la programación de audio bajo sistemas GNU/Linux.
\item Entender las bases del análisis de sonidos en el dominio de la
  frecuencia. 
\item Utilizar un enfoque de análisis, diseño y codificación orientado
  a objetos, de una forma lo más clara y modular posible, para
  permitir ampliaciones y modificaciones sobre la aplicación por
  terceras personas.
\item Hacer uso de herramientas básicas en el desarrollo de software,
  como son los Sistemas de Control de Versiones para llevar
  un control realista del desarrollo del software, así como hacer de
  las veces de sistema de copias de seguridad.

\end{itemize}

\section{Planificación}

\section{Descripción general}

\section{Implementación}

\section{Conclusiones y difusión}



\end{document}
