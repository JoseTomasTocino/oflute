\section{Contexto y motivación}
Las nuevas tecnologías van filtrándose gradualmente en los centros
educativos, y las técnicas de enseñanza se están adaptando a las
opciones que ofrecen. El reparto de ordenadores portátiles a los
alumnos andaluces de 5º y 6º de primaria, dentro del marco de la
Escuela TIC 2.0, es buena muestra de ello. 

Por otro lado, las nuevas generaciones están en plena simbiosis con las
tecnologías de la información, cada vez más acostumbradas al empleo de
dispositivos electrónicos interactivos, y su uso ya les es prácticamente
instintivo. Por tanto, es beneficioso buscar nuevos métodos educativos que hagan
uso de las nuevas tecnologías.

En la búsqueda de materias educativas en las que aplicar el uso de las nuevas
tecnologías, la música, parte fundamental del programa curricular en la
educación primaria, ofrece una gran variedad de aspectos que podrían
desarrollarse utilizando tecnologías de la información. Es ahí donde este
proyecto hace su aportación, en la flauta dulce, un instrumento económico y
fácil de aprender que se usa tradicionalmente en la educación musical
obligatoria en España.

\section{Objetivos}
\label{sec:objetivos}

A la hora de definir los objetivos de un sistema, podemos agruparlos
en dos tipos diferentes: \textbf{funcionales} y
\textbf{transversales}. Los primeros se refieren a \textit{qué} debe
hacer la aplicación que vamos a desarrollar, e inciden
directamente en la experiencia del usuario y de potenciales
desarrolladores.

Por otro lado, los objetivos transversales son aquellos invisibles al
usuario final, pero que de forma inherente actúan sobre el resultado
final de la aplicación y sobre la experiencia de desarrollo de la misma.

\subsection{Funcionales}
\begin{itemize}
\item Crear un módulo de análisis del sonido en el dominio de la frecuencia para
  poder identificar las notas emitidas por una flauta dulce y capturadas
  mediante un micrófono en tiempo real.
\item Crear una aplicación de usuario que identifique y muestre en pantalla las
  notas que toca el usuario con la flauta dulce en cada momento.
\item Reutilizar el módulo de análisis en un juego en el que el
  usuario debe tocar correctamente las notas que aparecen en pantalla
  siguiendo un pentagrama.
\item Incluir un sistema de lecciones multimedia individuales que
  sirvan al alumno de referencia y fuente de aprendizaje.
\item Potenciar el uso de interfaces de usuario amigables, con un
  sistema avanzado de animaciones que proporcione un aspecto fluido y
  evite saltos bruscos entre secciones.
\end{itemize}

\subsection{Transversales}
\begin{itemize}
\item Obtener una base teórica sobre cómo se representa y caracteriza
  digitalmente el sonido.
\item Conocer las bases del \ac{DSP}, y su uso en aplicaciones de
  reconocimiento básico de sonidos, tales como sintonizadores y
  afinadores de instrumentos.
\item Adquirir soltura en la programación de audio bajo sistemas GNU/Linux.
\item Entender las bases del análisis de sonidos en el dominio de la
  frecuencia. 
\item Utilizar un enfoque de análisis, diseño y codificación orientado
  a objetos, de una forma lo más clara y modular posible, para
  permitir ampliaciones y modificaciones sobre la aplicación por
  terceras personas.
\item Hacer uso de herramientas básicas en el desarrollo de software,
  como son los \textbf{Sistemas de Control de Versiones} para llevar
  un control realista del desarrollo del software, así como hacer de
  las veces de sistema de copias de seguridad.
\end{itemize}


\section{Alcance}
\textbf{oFlute} se modela como una herramienta lúdico-educativa para
alumnos que comiencen a aprender a usar la flauta dulce,
proporcionando un entorno atractivo y ameno para el estudiante. Éstos
tendrán la posibilidad de recorrer una serie de pequeñas lecciones
sobre música en general, y el uso de la flauta dulce en particular.

Además, el usuario tendrá la posibilidad de comprobar sus
conocimientos sobre el uso de la flauta practicando, gracias a las
secciones de análisis de notas y de canciones, en las que la
aplicación valorará la pericia del estudiante con la flauta.  

\subsection{Limitaciones del proyecto}
El proyecto se limita al uso de la flauta dulce y no a otros
instrumentos por la enorme variabilidad de timbre entre ellos, lo que
supondría un enorme esfuerzo a la hora de generalizar el analizador de
frecuencias.

El sistema de lecciones se basa en plantillas XML en las que es
posible definir imágenes y texto para formar una pantalla
de información. En un futuro se ampliará para incluir otros elementos
multimedia así como lecciones con varias pantallas consecutivas.

Los sistemas de audio son una de las áreas en las que menos consenso hay entre
plataformas informáticas, por lo que la transportabilidad de las aplicaciones
suele ser compleja. El presente proyecto utiliza la API Simple de PulseAudio
como subsistema de sonido, que es en teoría compatible con plataformas Win32,
pero en la práctica su complejidad hace prácticamente inviable la portabilidad
de la aplicación.

\subsection{Licencia}
El proyecto está publicado como software libre bajo la licencia
\ac{GPL} versión 3. El conjunto de bibliotecas y módulos utilizados
tienen las siguientes licencias:
\begin{itemize}

\item A lo largo del proyecto se utilizan diferentes partes de las
  bibliotecas \textbf{Boost}~\cite{boost}, que utilizan la licencia
  \textit{Boost Software License}~\footnote{\url{http://www.boost.org/LICENSE_1_0.txt}}.
  Se trata de una licencia de software libre, compatible con la GPL, y
  comparable en permisividad a las licencias BSD y MIT.

\item \textbf{Gosu}~\cite{gosu}, la biblioteca de desarrollo de
  videojuegos que ha proporcionado el subsistema gráfico, utiliza la
  licencia \ac{MIT}. Cuando se compila en sistemas Windows, utiliza la
  biblioteca FMOD que es gratuita pero de código cerrado; en sistemas
  GNU/Linux, utiliza SDL\_mixer, que utiliza la licencia \ac{LGPL}.

\item \textbf{Kiss FFT}~\cite{kissfft}, la biblioteca utilizada para
  hacer el análisis de frecuencias, utiliza una licencia \ac{BSD}.

\item \textbf{PugiXML}~\cite{pugixml}, biblioteca de procesamiento de
  ficheros XML, se distribuye bajo la licencia MIT.

\item \textbf{PulseAudio}~\cite{pulseaudio} utiliza una licencia LGPL 2.1.
\end{itemize}

\section{Estructura del documento}
El presente documento se rige según la siguiente estructura:

\begin{itemize}
\item \textbf{\nameref{chap:introduccion}}. Se exponen las motivaciones y
  objetivos detrás del proyecto \textbf{oFlute}, así como información sobre las
  licencias de sus componentes, glosario y estructura del documento.
\item \textbf{\nameref{chap:calendario}}, donde se explica la planificación del
  proyecto, la división de sus etapas, la extensión de las etapas a lo largo del
  tiempo y los porcentajes de esfuerzo.
\item \textbf{\nameref{chap:fundamentos}}, que explica las labores de
  documentación y experimentación previas al desarrollo, que han servido para
  labrar una base de conocimientos que nos diera las suficientes garantías para
  afrontar el proyecto.
\item \textbf{\nameref{chap:analisis}}. Se detalla la fase de análisis del
  sistema, explicando los requisitos funcionales del sistema, los diferentes
  casos de uso, así como las principales operaciones con sus diagramas de
  secuencia y contratos.
\item \textbf{\nameref{chap:diseno}}. Seguido del análisis, se expone en detalle
  la etapa de diseño del sistema, con los diagramas de clases.
\item \textbf{\nameref{chap:implementacion}}. Una vez analizado el sistema y
  definido su diseño, en esta parte se detallan las decisiones de implementación
  más relevantes que tuvieron lugar durante el desarrollo del proyecto.
\item \textbf{\nameref{chap:pruebas}}. Listamos y describimos las pruebas que se
  han llevado a cabo sobre el proyecto para garantizar su fiabilidad y
  consistencia.
\item \textbf{\nameref{chap:conclusiones}}. Comentamos las conclusiones a las
  que se han llegado durante el transcurso y al término del proyecto.
\end{itemize}

Y los siguientes apéndices:
\begin{itemize}
\item \textbf{\nameref{chap:herramientas}}, donde detallamos qué hemos usado para la
  elaboración del proyecto.
\item \textbf{\nameref{chap:manual_instalacion}} del proyecto en sistemas nuevos.
\item \textbf{\nameref{chap:manual_usuario}}, donde se explica cómo usar la aplicación.
\item \textbf{\nameref{chap:manual_canciones}} a \textbf{oFlute}.
\item \textbf{\nameref{chap:manual_lecciones}} a \textbf{oFlute}.
\item \textbf{\nameref{chap:tutorial_gettext}}, utilizado a lo
  largo del proyecto.
\end{itemize}

Tras una revisión del calendario seguido, detallaremos a lo largo del
resto de la memoria el proceso de análisis, diseño, codificación y
pruebas que se siguió al realizar el proyecto.  

Los manuales de usuario y de instalación se incluyen tras un resumen
de los aspectos más destacables de proyecto y las conclusiones. En
dicho manual, se hallan dos apartados dirigidos a la ampliación de la
aplicación mediante la creación de nuevas lecciones y de nuevas
canciones, respectivamente.

\subsection{Acrónimos}

%%
%% acronimos.tex
%% 
%% Made by Antonio García
%% Login   <antonio@localhost>
%% $Id: acronimos.tex 626 2008-07-05 00:06:02Z antonio $
%%

\begin{acronym}
  \acro{ABI}{Application Binary Interface}
  \acro{ACL2s}{ACL2 Sedan}
  \acro{ACL2}{A Computational Logic for Applicative Common Lisp}
  \acro{AMD}{Advanced Micro Devices}
  \acro{API}{Application Programming Interface}
  \acro{APL}{Apache Public License}
  \acro{ASCII}{American Standard Code for Information Interchange}
  \acro{BNF}{Backus-Naur Form}
  \acro{BOM}{Byte Order Mark}
  \acro{BSD}{Berkeley Software Distribution}
  \acro{C3}{Chrysler Comprehensive Compensation System}
  \acro{CAP}{Complex Arithmetic Processor}
  \acro{CASE}{Computer Assisted Software Engineering}
  \acro{CPAN}{Comprehensive Perl Archive Network}
  \acro{CPU}{Central Processing Unit}
  \acro{CSS}{Cascading Style Sheet(s)}
  \acro{CVS}{Concurrent Versions System}
  \acro{DOM}{Document Object Model}
  \acro{DSP}{Digital Signal Processing}
  \acro{DTD}{Do\-cu\-ment Type De\-fi\-ni\-tion}
  \acro{EAFP}{Easier to Ask for Forgiveness than Permission}
  \acro{FDL}{Free Documentation License}
  \acro{FSF}{Free Software Foundation}
  \acro{GCC}{GNU Compiler Collection}
  \acro{GCJ}{GNU Compiler for Java}
  \acro{GCL}{GNU Common Lisp}
  \acro{GIMP}{\acs{GNU} Image Manipulation Program}
  \acro{GML}{Generalized Markup Language}
  \acro{GNU}{GNU is Not Unix}
  \acro{GPL}{General Public License}
  \acro{GTK+}{\acs{GIMP} Toolkit}
  \acro{GUI}{Graphical User Interface}
  \acro{HTML}{Hyper Text Markup Language}
  \acro{IBM}{International Business Machines}
  \acro{IDE}{Integrated Development Environment}
  \acro{IEEE}{Institute of Electrical and Electronics Engineers}
  \acro{IIS}{Internet Information Server}
  \acro{J2SE}{Java 2 Standard Edition}
  \acro{JAXP}{Java API for XML Parsing}
  \acro{JDBC}{Java DataBase Connectivity}
  \acro{JDK}{Java Development Kit}
  \acro{JIT}{Just In Time}
  \acro{JRE}{Java Runtime Environment}
  \acro{JSON}{JavaScript Object Notation}
  \acro{JSR}{Java Specification Request}
  \acro{JVM}{Java Virtual Machine}
  \acro{LGPL}{Lesser General Public License}
  \acro{LTS}{Long Time Support}
  \acro{MDI}{Multiple Document Interface}
  \acro{MIT}{Massachusetts Institute of Technology}
  \acro{MVC}{Model View Controller}
  \acro{MVP}{Model View Presenter}
  \acro{ODF}{Open Document Format}
  \acro{PAR}{Perl ARchive Toolkit}
  \acro{PDF}{Portable Document Format}
  \acro{POD}{Plain Old Do\-cu\-men\-ta\-tion}
  \acro{POSIX}{Portable Operating System Interface, UniX based}
  \acro{Perl}{Practical Extraction and Report Language}
  \acro{RPM}{RPM Package Manager}
  \acro{RUP}{Rational Unified Process}
  \acro{SAX}{Simple API for XML}
  \acro{SDI}{Single Document Interface}
  \acro{SGML}{Standard Generalized Markup Language}
  \acro{SVG}{Structured Vector Graphics}
  \acro{SWT}{Standard Widget Toolkit}
  \acro{TCK}{Technology Compatibility Kit}
  \acro{TDI}{Tabbed Document Interface}
  \acro{UML}{Unified Modelling Language}
  \acro{URI}{Uniform Resource Identifier}
  \acro{URL}{Uniform Resource Locator}
  \acro{UTF}{Universal Transformation Format}
  \acro{W3C}{World Wide Web Consortium}
  \acro{WWW}{World Wide Web}
  \acro{XHTML}{eXtensible Hyper Text Markup Language}
  \acro{XML}{eXtensible Markup Language}
  \acro{XP}{eXtreme Programming}
  \acro{XSL-FO}{eXtensible Stylesheet Language Formatting Objects}
  \acro{XSLT}{eXtensible Stylesheet Language Transformations}
  \acro{XSL}{eXtensible Stylesheet Language}
  \acro{YAML}{YAML Ain't a Markup Language}
  \acro{YAXML}{YAML XML binding}
\end{acronym}

%%% Local Variables: 
%%% mode: latex
%%% TeX-master: "../memoria"
%%% End: 



%%% Local Variables: 
%%% mode: latex
%%% TeX-master: "../memoria"
%%% End: 
