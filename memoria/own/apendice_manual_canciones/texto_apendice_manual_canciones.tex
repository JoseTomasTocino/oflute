En \textbf{oFlute} existe la posibilidad de añadir nuevas canciones al juego, de
forma que los usuarios puedan ampliar el juego y que éste se convierta en una
herramienta lúdica prácticamente ilímitada.

Así, en un futuro podrán existir, por ejemplo, packs temáticos de canciones
descargables desde internet, con los que seguir practicando nuestra pericia con
la flauta dulce.

\section{Ficheros necesarios}

Cada una de las canciones presentes en el juego necesitará un fichero donde se
definirán las características y notas de la canción. El nombre de estos ficheros
sigue la estructura \texttt{songN.xml}, donde \textit{N} es el número de canción
a añadir.

Los ficheros de las canciones se encuentran en la carpeta \texttt{canciones},
ubicada en el directorio raíz de la instalación de oFlute. 

\section{Estructura del fichero de canciones}

\subsection{Lenguaje XML}

Cada fichero de canción es un archivo \textit{XML}. \marginpar{REFERENCIA} XML
son las siglas, en inglés, de \textit{eXtensible Markup Language} -- lenguaje de
marcado extensible. La tecnología XML busca dar solución al problema de expresar
información estructurada de la manera más abstracta y reutilizable posible. 

Para ello, utiliza una serie de \textit{etiquetas}, organizadas de forma
jerárquica, que siguen la forma \texttt{<nombre>}, donde \textit{nombre} es la
identificación del elemento que se está representando. Las etiquetas deben
cerrarse utilizando la sintaxis \texttt{</nombre>}. Entre las etiquetas de
apertura y cierre pueden anidarse otras etiquetas, así como información en
formato texto.

Todo documento XML tiene una etiqueta \textit{raíz}, de la cual cuelgan todas
las demás. En nuestro caso, los ficheros que representan las canciones tendrán
como raíz el elemento \texttt{Song}. Este es un ejemplo de un fichero completo:



\begin{minted}{xml}
<?xml version="1.0" ?>
<Song>
  <Title>Los pajaritos</Title>
  <Desc>Van por el aire</Desc>
  <BPM>60</BPM>
  <Notes>do5n mi5c fa5c sol5n sol5c sol5c la5n la5c la5c sol5n 
   sol5n fa5n fa5n mi5n mi5n re5c mi5c fa5c re5c do5n do5n</Notes>
</Song>
\end{minted}

\subsection{Campos iniciales}

\subsubsection{Título de la canción}

%\usestyle{default}

El título de la canción se indicará mediante la etiqueta \texttt{<Title>}, por ejemplo:

\inputminted{xml}{apendice_manual_canciones/snippet_1}
