En este capítulo se detallan las pruebas a las que se ha sometido
\textbf{oFlute}. Habitualmente, probar un videojuego o aplicación con gran
interactividad suele ser un proceso complejo. Dada la gran cantidad de
situaciones distintas que pueden tener lugar, es difícil recrear y probar todos
los escenarios posibles. Tanto es así que, en los grandes equipos de desarrollo,
hay empleados que se dedican al completo a probar los productos.

En oFlute, las pruebas se han organizado al final de cada iteración, comprobando
el correcto funcionamiento de cada nueva funcionalidad, por separado, y luego
integrando el conjunto. Además, ciertos módulos no asociados a iteraciones (como
el sistema de animaciones (sección \ref{sec:animaciones}) también fueron
probados de manera independiente.

\section{Pruebas unitarias}
Las \textbf{pruebas unitarias} se encargan de comprobar el correcto
funcionamiento de módulos y fragmentos por separado. En nuestro caso, todos los
módulos fueron probados por separado, utilizando a tal efecto ramas de
desarrollo alternativas que probaban cada funcionalidad en particular. 

\section{Pruebas de integración}

\section{Pruebas de jugabilidad}

\section{Pruebas de interfaz}

