\documentclass[landscape]{article}

\usepackage[utf8]{inputenc}
\usepackage[spanish]{babel}

\usepackage[top=1cm, bottom=1cm, left=1cm, right=1cm]{geometry}

\usepackage{multicol}
\setlength{\columnsep}{20pt}

\renewcommand{\familydefault}{\sfdefault}

\usepackage{setspace}
\onehalfspacing

\setlength{\parskip}{0.4cm}

\title{Apuntes para la presentación}
\author{José Tomás Tocino García}
\date{}

\usepackage{framed}

\newenvironment{nota}
{% This is the begin code
\begin{framed} \noindent\itshape
}
{% This is the end code
\end{framed}\vspace{-0.5cm} }

\begin{document}

\pagestyle{empty}
\begin{multicols*}{2}

\begin{center}
  \begin{Large}\textbf{Apuntes para la presentación del PFC}\end{Large}\\
  \begin{large}José Tomás Tocino García\end{large}\\[0.1cm]
\end{center}

Buenos días, mi nombre es José Tomás Tocino García, soy alumno de Ingeniería
Técnica en Informática de Sistemas, y voy a presentar mi proyecto titulado
``\textit{oFlute: reconocimiento de señales aplicado al aprendizaje de la flauta
  dulce}''.

\begin{nota}
  Pasar a transparencia 2: índice.
\end{nota}

Este es el índice que vamos a seguir durante la presentación. \textbf{EXPLICAR
  ÍNDICE}

\begin{nota}
  Contexto social.
\end{nota}

En la actualidad, las nuevas generaciones están en plena simbiosis con las
tecnologías de la información. Ya sea mediante redes sociales, videojuegos o
cualquier otra clase de sistema multimedia, desde muy jóvenes se acostumbran al
empleo de dispositivos electrónicos, haciendo que su uso sea prácticamente
instintivo.

\framebox{STEP} Por otro lado, las nuevas tecnologías van filtrándose gradualmente en los
centros educativos. Muestra de elo es el reparto de ordenadores portátiles a los
alumnos andaluces de 5o y 6o de primaria, dentro del marco de la Escuela TIC
2.0, que ha llevado a cabo la Junta de Andalucía.

\framebox{STEP} Estos dos factores favorecen la aparición de nuevas técnicas docentes, basadas
en el uso de recursos multimedia y equipos informáticos. Estas técnicas tienen
una doble ventaja. Por un lado, facilitan a los profesores el impartir su
temario, y por otro lado, resultan atractivas para los alumnos, que a menudo
muestran más interés y adquieren más fácilmente el conocimiento en comparación
con las técnicas tradicionales.

\begin{nota}
  Concepción del proyecto
\end{nota}

Por ello, se tomó la decisión de desarrollar un videojuego educativo, que
pudiera utilizarse fácilmente en las escuelas y que tuviera una utilidad real
con respecto al aprendizaje de los alumnos. Actualmente se trata de un sector
innovador, en pleno crecimiento y con vistas a un futuro prometedor.

\framebox{STEP} La primera cuestión que surgió fue el tema del proyecto. ¿Sobre qué aspecto
educativo debería versar la aplicación? ¿Qué asignatura se beneficiaría del
desarrollo?

Estuvimos pensando en las diferentes materias que se imparten en los niveles de
primaria, viendo cuál de ellas podría beneficiarse más de un proyecto de esta
clase, y finalmente se decidió orientar la aplicación hacia la música.

\framebox{STEP} Dentro de la música había bastantes aspectos que podrían servir,
pero el que nos resultó más atractivo fue el \framebox{STEP} aprendizaje de la
flauta dulce. Es un instrumento económico y fácil de aprender que siempre se ha
utilizado en las clases de música, por lo que está al alcance de la inmensa
mayoría de alumnos.

\begin{nota}
  Objetivos.
\end{nota}

Así pues, una vez que decidimos la temática de la aplicación, nos planteamos una
serie de objetivos para el proyecto. El primer objetivo, que fue uno de los más
importantes, fue adquirir una base de conocimientos que me permitiera llevar a
cabo el proyecto. Yo no tenía conocimiento previo alguno sobre programación de
sistemas de sonido ni sobre análisis y procesamiento de señales. Es una temática
que nunca había abordado, ni durante la carrera ni de forma independiente.

Una vez adquiridos los conocimientos, \framebox{STEP} el siguiente paso
fundamental era crear un módulo de análisis de sonido que permitiera reconocer
las notas tocadas por la flauta. Este módulo es uno de los componentes
principales de la aplicación, y realmente el que ha supuesto más trabajo. Al fin
y al cabo, la viabilidad de la aplicación dependía del funcionamiento correcto
de este módulo. 

\framebox{STEP} Otro de los objetivos fue crear un sistema de interpretación de
canciones, basado en el módulo de análisis, que debería presentar al jugador una
canción en pantalla, en un pentagrama, para que el jugador la interprete.

\framebox{STEP} Además, el proyecto cuenta con un sistema de lecciones multimedia, totalmente
ampliable por el usuario, que sirve al estudiante como fuente de conocimiento
sobre música en general, y el uso de la flauta dulce en particular.

\framebox{STEP} Finalmente, se impuso como objetivo potenciar el uso de interfaces de usuario
amigables, con un sistema de animaciones que proporcione un aspecto fluido y
evite saltos bruscos entre secciones

\begin{nota}
  Motivaciones personales.
\end{nota}

Además de los objetivos sobre el proyecto, hay varias motivaciones personales
que me impulsaron al desarrollo de la aplicación. En primer lugar, quise conocer
cómo se representa un sonido digitalmente, así como adquirir soltura en la
programación relacionada con el audio.

\framebox{STEP} Por otro lado, quería conocer las bases del DSP, al menos a un
nivel que me permitiera enfrentarme al proyecto con garantías. En particular,
entender lo básico de las técnicas de análisis de audio en el dominio de la
frecuencia.

\framebox{STEP} También tenía interés en ampliar mis conocimientos sobre
desarrollo de videojuegos, que anteriormente había adquirido durante el
desarrollo de otros proyectos. Además, la variedad del proyecto daría lugar al
uso de tecnologías que resultaría interesante conocer. Por último, me gustaba la
idea de aportar un proyecto de software a al comunidad, sobre todo con vistas a
ser incluido en Guadalinex, lo que resultaría en su uso en los portátiles de los
alumnos de primaria, que son el perfecto público objetivo.

\begin{nota}
  oFlute.
\end{nota}

De esta idea nace oFlute. oFlute es una herramienta lúdico-educativa, que ayuda
al aprendizaje de la flauta dulce, proporcionando un entorno atractivo y ameno
para el estudiante. Éste interactuaár con la aplicación utilizando la flauta
dulce, a través de varias funcionalidades de la aplicación que a continuación
explicaremos.

\begin{nota}
  Analizador de notas.
\end{nota}

Primero tenemos el analizador de notas. Nos permite analizar las notas que
tocamos con la flauta y que el micrófono captura, viendo en cada momento qué
nota estamos tocando. Esto nos puede servir, por ejemplo, para ver si estamos
tocando correctamente una nota, o para saber a qué nota corresponde cierta
posición de los dedos.

\begin{nota}
  Motor de lecciones.
\end{nota}

Seguidamente tenemos el motor de lecciones, capaz de cargar lecciones multimedia
de forma dinámica, incluyendo animaciones, y es totalmente ampliable de forma
fácil.

\begin{nota}
  Motor de canciones.
\end{nota}

Por último tenemos el sistema de canciones, que también es completamente
ampliable, y que nos permitirá interpretar canciones con la flauta y que el
sistema nos vaya diciendo cómo lo estamos haciendo, resultando en una puntuación
final.

\begin{nota}
  Planificación.
\end{nota}
Durante la fase de demostración veremos cada una de estas secciones con más detalle.

Pasando al calendario y la planificación del proyecto, seguimos un modelo de
desarrollo iterativo, que tuvo cinco fases bien diferenciadas.

\framebox{STEP} El primer paso fue la adquisición de la base de
conocimientos. Fue una etapa larga en la que se consultó bastante documentación,
principalmente online, también listas de correos de newsgroups, y se hicieron
algunas pruebas de concepto. También en esta etapa se revisaron las posibles
herramientas a utilizar, bibliotecas y demás, y, aunque no existen juegos con la
misma temática, se consultaron juegos de temática similar para coger ideas.

\framebox{STEP} En la siguiente iteración se desarrolló el analizador básico,
que presentaremos posteriormente, y que sirvió como base para el resto de la
aplicación.

\framebox{STEP} La tercera iteración se dedicó al diseño y desarrollo de la
interfaz gráfica de usuario. En esta etapa se diseñaron el logo, las secciones,
y se implementó un sistema de animaciones que detallaremos en la siguiente sección.

\framebox{STEP} En la cuarta iteración se trabajó en el sistema de lecciones
multimedia, y \framebox{STEP} en la quinta y última se desarrolló el motor de
canciones.

\begin{nota}
  Diagrama de Gantt
\end{nota}

Éste es el diagrama de Gantt en el que se pueden distinguir las distintas
etapas del proyecto.

\begin{nota}
  
\end{nota}

Ahora vamos a hablar de los detalles de implementación de algunas de las partes
del proyecto que hayan resultado más interesantes. En primer lugar vamos a
hablar de cómo se desarro


 \vfill \pagebreak


\end{multicols*}
\end{document}